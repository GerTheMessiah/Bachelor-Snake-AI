\chapter{Vorgehen} \label{chap:Vorgehen}
In diesem Kapitel soll das weitere Vorgehen thematisiert werden. Dazu erfolgt eine genaue Erklärung der weiteren Schritte.\\
Angestammtes Ziel dieser Ausarbeitung ist es eine Methodik zu finden, mit welcher man einen möglichst optimalen Agenten bestimmen kann. Dabei lässt das Wort optimal viel Spielraum für Evaluationskriterien. Im Kapitel \ref{chap:Anforderungen} wurden daher Anforderungen definiert, nach welchen evaluiert wird. Mit Hilfe von Optimierungen, welcher in Kapitel \ref{chap:Verwandte_Arbeiten} herausgearbeitet und konkretisiert wurden, soll die Leistungsfähigkeit der Agenten in den verschiedenen Evaluationskriterien erhöht werden. Weitere folgen im Kapitel \ref{chap:Optimierungen}. Um jedoch einen einen Vergleich durchführen zu können wird ein Basis (Baseline) benötigt, auf welcher verglichen wird.

\section{Bemerkung zur Netzstruktur}
Das Editieren der Netzstruktur wird in dieser Ausarbeitung nicht als Optimierung in Erwägung gezogen. Veränderungen der Netzstruktur haben weitreichende Folgen auf die Obs. und einige Hyperparameter, wie z.B. die Lernrate, welche bei Veränderung der Netzstruktur ebenfalls angepasst werden sollte. In dieser Ausarbeitung wird daher nur mit der Netzstruktur, welche unter \ref{sec:Impl_Netzstruktur} vorgestellt wurde.
Ein Vergleich unterschiedlicher Netzstrukturen kann daher im Vorhinein unter den nicht optimierten Agenten geschehen.

\section{Baseline Datenerhebung, Vergleich und Evaluation}
Zu Erstellung grundlegender Vergleichsdaten sollen die in Kapitel \ref{chap:Agenten} vorgestellten Agenten in zwei Trainingsdurchläufe untersucht werden. Die erhaltenden Daten bzw. trainierten Agenten werden dann unter den in \ref{sec:Anforderungen_an_die_Evaluation} definierten Anforderungen evaluiert und vergleichen.\\
\\Um die Menge an Agenten zu minimieren und damit Konvergenz herbeizuführen, werden die zwei schlechtesten Agenten für ein Anforderung, wie z.B. Performance, Effizienz, Robustheit usw., disqualifiziert. Dies geschieht auf Basis der Baseline Daten.

\section{Anwendung der Optimierungen}
Nach dem Baseline Vergleich und Aussortierungen der schlechtesten Agenten sollen diese nun optimiert werden.
Dazu werden vier Optimierungen auf die einzelnen Agenten angewendet, welche im Kapitel \ref{chap:Optimierungen} näher thematisiert werden. 
Anschließend erfolgt eine Datenerhebung nach den Anforderungen \ref{label} der optimierten Agenten, zuzüglich deines Vergleiches und einer Evaluation.\\
Nach dieser ist der optimale Agent mit der für jede Anforderung gefunden sein.