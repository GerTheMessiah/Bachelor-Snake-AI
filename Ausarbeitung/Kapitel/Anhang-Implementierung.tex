\chapter{Anhang zur Implementierung}

\section{AV-Network} \label{sec:Anhang_AV_Network}
Mithilfe der Module, kann eine neues NN-Element erstellt werden.
Es wurde daher die Klasse AV\_NET definiert, welche zugleich als NN-Element benutzt werden kann.
\begin{lstlisting}[caption=Implementierung des AV-NET, style=Python]
class AV_NET(nn.Module):
	def __init__(self):
		super(AV_NET, self).__init__()
		self.AV_NET = nn.Sequential(
		nn.Conv2d(in_channels=6, out_channels=8, \
		kernel_size=(3, 3), stride=(1, 1), padding=(1, 1)),
		nn.ReLU(),
		nn.Conv2d(in_channels=8, out_channels=8, \
		 kernel_size=(3, 3), stride=(1, 1), padding=(1, 1)),
		nn.ReLU(),
		nn.ZeroPad2d((0, 1, 0, 1)),
		nn.MaxPool2d(kernel_size=2, stride=2),
		nn.Flatten(),
		nn.Linear(392, 256),
		nn.ReLU(),
		nn.Linear(256, 128)
		)
	def forward(self, av):
		return self.AV_NET(av)
\end{lstlisting}
Diese besteht aus zwei Convolutional (Conv2d), einem Max-Pooling und zwei Fully Connected Layers (Linear). 
Zwischen den Layers finden weitere Transformationen statt, welche im \autoref{subsec:Konzept_Netzstruktur} erklärt wurden.
Jedes NN-Element muss zwingend eine forward Methode implementieren, um die Propagierung der Input-Daten durch das Netzwerk zu steuern. Aus diesem Grund existiert eine forward Methode für diese Klasse, welche die AV (around\_view) durch die dargestellten NN-Schichten und Funktionen propagiert. Das Ergebnis wird zurückgegeben.

