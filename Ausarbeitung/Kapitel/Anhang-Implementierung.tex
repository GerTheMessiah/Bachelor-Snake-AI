\chapter{Anhang zur Implementierung}

\section{Around-View} \label{sec:Anhang-Implementierung-Around-View}
Zu Erstellung der around\_view (AV) werden zunächst einige Konstanten gesetzt bzw. berechnet, wie z.B. die width oder c\_s und c\_h erstellt. Erstere entspricht dem Radius des Ausschnittes um den Kopf. Die letzteren sind Farbkonstanten, welche im Player definiert wurden (siehe \ref{sec:Implementierung_Player}).\\
Danach wird über jeden Eintrag, welcher im Ausschnitt der AV liegt, iteriert. Dieser Ausschnitt besitzt die Form eines Vierecks in dessen Mittelpunkt sich der Kopf der Snake befindet. Für jeden dieser Einträge wird überprüft, ob er einem der Merkmalen aus \ref{tab:around_view} entspricht. Sollte ein Merkmal erkannt worden sein, wie z.B. die Position des Apfels, so wird diese in den dazugehörigen Channel eingepflegt.
\begin{python}
	def create_around_view(pos, id, g):
		width = 6
		c_h = id * 2
		c_s = id
		tmp_arr = np.zeros((6, width * 2 + 1, width * 2 + 1), dtype=np.float64)
	
		for row in range(-width, width + 1):
			for column in range(-width, width + 1):
				if on_playground(pos[0] + row, pos[1] + column, g.shape):
					a, b = pos[0] + row, pos[1] + column
					if g[a, b] == c_s:
						tmp_arr[1, row + width, column + width] = 1
						continue
	
					elif g[a, b] == c_h:
						tmp_arr[2, row + width, column + width] = 1
						continue
	
					elif g[a, b] == 0:
						tmp_arr[3, row + width, column + width] = 1
						continue
	
					elif g[a, b] == -1:  # End of snake tail.
						tmp_arr[4, row + width, column + width] = 1
						continue
	
					elif g[a, b] == -2:
						tmp_arr[5, row + width, column + width] = 1
	
				else:
				tmp_arr[0, row + width, column + width] = 1
	
		return np.expand_dims(tmp_arr, axis=0)
\end{python}

\section{Distanzbestimmung} \label{sec:Anhang-Implementierung-Distanzen}
Ein wesentlicher Teil der scalar\_obs (SO) besteht aus den Distanzen, welche von der create\_distances Funktion generiert werden. Zu Beginn wird ein Null-Array der Länge 24 (3 * 8 = 24), in welches die Distanzen eingetragen werden und eine List namens grad\_list erstellt. In dieser werden Faktoren gehalten, welche den Grad der Sucherlinie (siehe \ref{fig:Observation}) angeben. Dabei entsprechen die Faktoren den Himmels- bzw. Nebenhimmelsrichtungen in der folgenden Reihenfolge (0° = N, 45° = NO, 90° = O, 135° = SO, 180° = S, 225° = SW und 270° = W, 315° = NW).\\
Danach wird über die drei zu suchenden Objekte (Wände, Snake-Glieder, Apfel) iteriert, wobei für jedes dieser Objekt die dist Unterfunktion aufgerufen wird. Die Rückgaben werden in das Array eingepflegt und dieses ,sobald alle Distanzen bestimmt worden sind, zurückgegeben.
\begin{python}
	def create_distances(pos, ground):
		obs = np.zeros(24, dtype=np.float64)
		a = 0
		grad_list = [(-1, 0), (-1, 1), (0, 1), (1, 1), (1, 0), (1, -1), (0, -1), (-1, -1)]
		for wanted in [[], [-1, 1, 2], [-2]]:
			for grad in grad_list:
				obs[a] = dist(ground, pos, wanted, *grad)
				a += 1
		return obs
\end{python}
Der dist Unterfunktion werden das Spielfeld (ground), die Position des Snake-Kopfes (p\_pos) die zu suchenden Werte (wanted) und fac\_0 und fac\_1 übergeben. Diese stellen die oben genannten Faktoren dar, welcher die Ausrichtung der Suchlinien definiert.
Nach der Initialisierung weiterer Hilfsvariablen und der Distanz (dist\_), wird der erste Index generiert, welcher dem ersten Feld auf dem Weg der Linie entspricht. Sollte der Eintrag an dieser Stelle dem gesuchtem Objekt entsprechen, so wird die Zahl zwei zurückgegeben. Anderenfalls wird dist\_ inkrementiert und der nächste höhere Index der auf dem Weg der Linie liegt generiert. Dann wird wieder so verfahren wie oben bereits erwähnt.\\
Sollte einer der Indexe jedoch das Spielfeld verlassen, so wird eine Distanz von null zurückgegeben.\\
Diese Prozedur endet entweder mit der Rückgabe einer Distanz zum Objekt, falls dieses gefunden wird, oder durch Rückgabe von null falls das Objekt nicht im Pfad der Linie liegt. 
\begin{python}
	def dist(ground, p_pos, wanted_hit, fac_0, fac_1):
		dist_, i_0, i_1 = 0, 1, 1
		p_0 = p_pos[0] + fac_0 * i_0
		p_1 = p_pos[1] + fac_1 * i_1
		while on_playground(p_0, p_1, ground.shape) and ground[p_0, p_1] not in wanted_hit:
			i_0 += 1
			i_1 += 1
			dist_ += 1
			p_0 = p_pos[0] + fac_0 * i_0
			p_1 = p_pos[1] + fac_1 * i_1
		if not on_playground(p_0, p_1, ground.shape) and bool(wanted_hit):
			return 0
		return 1 / dist_ if dist_ != 0 else 2
\end{python} 


